\documentclass[10pt,a4paper]{article}

\usepackage{ifluatex}
\ifluatex
	\usepackage{fontspec}
	\setmainfont{Linux Libertine O}
\else
	\usepackage[utf8]{inputenc}
	\usepackage{libertine}
	\usepackage[T1]{fontenc}
\fi

\usepackage[ngerman,english]{babel}
\usepackage{amsmath}
\usepackage{amsfonts}
\usepackage{amssymb}

\usepackage{color}

\newcommand{\todo}[1]{ \colorbox{yellow}{\textcolor{red}{\textbf{TODO:} {#1}}}\\ }

\begin{document}

\tableofcontents


%englische begriffe dazu?
%mathe schreibweisen? dx/x


\section{Beweisführung}

% Impliziert => <=>


\section{Zahlen}
\begin{tabular}{ll}
$\mathbb{N}$ & Ganze Zahlen 0 bis $\infty$ \\
$\mathbb{N}^*$ & Ganze Zahlen 1 bis $\infty$ \\
$\mathbb{Z}$ & Ganze Zahlen $[-\infty, \infty]$ \\
$\mathbb{Q}$ & Rationale Zahlen $\frac{m}{n}$\\
$\mathbb{R}$ & Reelle Zahlen ($\mathbb{Q}$ + Irrationale Zahlen)\\
$\mathbb{C}$ & Komplexe Zahlen\\
\end{tabular} 

\todo{Weitere Erklärung zu Zahlengruppen -> Referenz auf Gruppen/Ringe/etc}

\section{Gruppen, Ringe, Körper}
\todo{...}

\subsection{Gruppen}
\todo{...}

\section{Komplexe Zahlen}

\renewcommand{\arraystretch}{1.5}
\begin{tabular}{ll}
Basis Definition & $ z = a + b \cdot i $ \\ 
				 & $ i^{2} = -1 $ \\ %wurzel -1 = i?
				 & $\sqrt{-1} = \sqrt{i^2} = i $ \\
				 & $Re(a + bi) = a$ \\
				 & $Im(a + bi) = b$ \\
Addition 		 & $ (a + bi) + (c + di) = (a+c)+(b+d)i $\\
Subtraktion 	 & $ (a + bi) - (c + di) = (a-c)+(b-d)i $\\
Multiplikation   &  $(a+b\,i)\cdot(c+d\,i)=(ac-bd) + (ad+bc)\cdot i. $\\
Division         & 
    $\frac{a+b\,\mathrm i}{c+d\,\mathrm i} = \frac{(a+b\,\mathrm i)(c-d\,\mathrm i)}{(c+d\,\mathrm i)(c-d\,\mathrm i)} = \frac{ac+bd}{c^2+d^2}+\frac{bc-ad}{c^2+d^2}\cdot\mathrm i. $\\
Betrag 			 & $ |z| = \sqrt{a^{2} + b^{2}} $ \\ 
\end{tabular} 

%mengen?
%logik? <-> -> etc

\section{Grund-Rechenregeln}

\subsection{Potenzen}

$ a^{b} * a^{c} = a^{b+c} $\\
$ \dfrac{1}{x^2} = x^{-2}$

\subsection{Wurzeln}

$\sqrt[n]{a} \cdot \sqrt[n]{b} = \sqrt[n]{a \cdot b}  $ \\
$\sqrt[n]{x^{m}} = x^{\frac{m}{n}}  $ \\
$\sqrt[n]{x} = x^{\frac{1}{n}} = exp(\dfrac{ln(x)}{n})$

\subsection{Brüche}

\todo{Addition}
\todo{Subtraction}
\todo{Multiplikation}
\todo{Division}

$ \dfrac{a^{n}}{b^{m}} = a^{n} \cdot b^{-m}$

\subsection{Summen}

Umformungen


\subsection{Fakultät}

$ n! = 1\cdot 2\cdot 3\cdot \dots\cdot n =  \prod \limits_{k=1}^{n}k $\\
$ n! = n * (n-1)!$\\
$ 0! = 1 $\\
$ 1! = 1 $\\

Umformungen

% n! =n *(n -1)! 
% \(\prod \limits_{i=1}^{n+1}i = 1\cdot 2\cdot\dots\cdot n\cdot (n+1) \)
%binomialkoeffizient


\subsection{e-Funktion/Logarithmus}

$ e = \lim_{n\to\infty} \left(1+\frac{1}{n}\right)^n $\\
$ e^{x} = exp(x) = \sum_{n=0}^\infty \dfrac{x^{n}}{n!} $ \\
$ e^{1} = \sum_{n=0}^\infty \dfrac{1}{n!} = 2.7182818284590...$\\
$ a^{x} = e^{x \cdot ln a}$\\
$ ln(e^y) = y $\\

\subsection{Trigonometrische Funktionen (sin/cos/tan)}

% bogenmaß
% umfang des einheitskreis -> 2\pi
% bogenmaß = [0, 2\pi]
% rad
% der grad ist dann 0-360 <-> 0 - 2\pi

% tabelle Grad-PI-rad sin/cos siehe script


%umformungen sinus auflösen etc

\subsection{Binominialkoeffizient}

$ \binom nk $
$ \binom nk = \prod_{j=1}^k \frac{n + 1 - j}j$

%rechenregeln

\section{Hilfsmittel}

\begin{itemize}
\item Binomische Formeln
\item Quadratische Ergänzung
\item Vollständige Induktion
\end{itemize}

% quadratische Ergänzung
% binomische formeln

%% einordnen von gleichungen?

\section{Differentialrechnung}

\subsection{Notation}
\begin{flushleft}

Eine Funktion $f \colon U \to \mathbb{R}$, die ein offenes Intervall $U$ in die reellen Zahlen abbildet, heißt differenzierbar an der Stelle $x_0 \in U$, falls der Grenzwert existiert.

\[\lim_{x\to x_0} \frac{f(x) - f(x_0)}{x - x_0} = \lim_{h\to 0} \frac{f(x_0 +h) - f(x_0)}{h}   (h = x - x_0)\]

%steigungsdreieck?

Dieser Grenzwert heißt Differentialquotient oder Ableitung von $f$ nach $x$ an der Stelle $x_0$ und wird als
$ f'(x_0)$   oder   $\left.\frac{\mathrm df(x)}{\mathrm dx}\right|_{x=x_0}$   oder   $\frac{\mathrm df}{\mathrm dx}(x_0)$   oder   $\frac{\mathrm d}{\mathrm dx}f(x_0)$   notiert 

Nicht überall differenzierbar:
$f(x) = |x|$

$f'(0)$ existiert nicht

\end{flushleft}
    
\subsection{Ableitungsregeln}


\begin{tabular}{ll}
Potenzregel & $\left(x^n\right)' = n x^{n-1} $ \\
Produktregel & $(g\cdot h)' = g' \cdot h + g \cdot h'$ \\
Quotientenregel & 
    $\left(\frac{g}{h}\right)' = \frac{g' \cdot h - g \cdot h'}{h^2}$ \\
Kettenregel & $(g \circ h)'(x) = (g(h(x)))' = g'(h(x))\cdot h'(x)$\\
\end{tabular}

% (\ln(f))' = \frac{f'}{f} 


\subsubsection{Nullstellen}


\todo{p/q-formel}
%nullstellen x² x³ -> p/q formel

$ x^2+px+q=0$\\
$x_{1,2} = - \frac{p}{2}\pm\sqrt{\left(\frac{p}2\right)^2 - q}$

%Kurveldiskussion
% sattelpunkt
% extrem punkt

% differenzierung ableitungsregeln

\subsection{Partielle Ableitungen}

Für Funktionen mit mehreren Argumenten:
$f(x,y) = x^2 \sin(xy)$

Für einzelne Argumente ableiten:
$\dfrac{\partial f}{\partial x_i}$

% jacobi matrix

%mittelwertsatz der differntialrechnung

% rechnen mit mehreren variablen


\section{Integralrechnung}

Umkehrschritt der Differentialrechnung
Aufleitung

Berechnet Fläche einer Funktion
[Bild]


\subsection{Stamm-Funktionen}

% https://www.youtube.com/watch?v=8ynwPeV_VGQ

%wurzel oder e funktion -> substituieren

% substitution
% https://www.youtube.com/watch?v=z_flzV7SBBw

%Partielle Integration

$\int f'(x) * g(x) dx = f(x)*g(x) - \int f(x) * g'(x) dx$

%partialbruch zerlegung

%mittelwertsatz der integralrechnung

% rechnen mit mehreren variablen

\section{Differentialgleichungen}

\subsection{Klassifikation}


\subsubsection{Ordnung}
	Die Ordnung entspricht der maximalen Ableitung

\subsubsection{Explizit/Implizit}
	Explizit: Die höchste Ableitung steht auf einer Seite alleine.
	
	
\subsubsection{Gewöhnliche/Partielle}
	Eine unabhänige variable ist gewöhnlich
	$f(x)$

	Mehrere Unabhängige Variablen?
	$f(x,y)$
	Partielle Ableitungen
	
	
\subsubsection{Linear/Nicht linear}

	$sin(x) \cdot y'' + 42 \cdot y' - x^2 \cdot y = $
	
	$ X \cot y $
	y einfach, nicht als Quadrat/im Sinus/Unter Wurzel/ etc
	
	X funktionen von unabhängigenvariable x, konstante, ...
	
	(imhomoginität darf von X abhängen)
	(-> Lineare Gleichungssysteme)
	
	Mit Konstanten-Koeffizienten
	
	
	Homogene Lineare 
	ist was anderes als generell homogen/inhomogen
	



Varianten:
	(DGL)
	einfache Differentialgleichungen
	partielle Differentialgleichungen

	homogone autonome
	
	lineare
	
Schreibweisen:
	$ f'(x) = 3f(x) + x^2 $
	$ y'(x) = 3y(x) + x^2 $
	

\section{Vektoren}

\subsection{Notation}
$\mathbb{R}^{n}$
$\vec{v}$

\subsection{Rechenregeln}
%rechnen add/sub/mul/div
% 

\section{Matrizen}

\subsection{Notation}

$\mathbb{R}^{m\times n}$

$A = \begin{pmatrix} a & b & c \\ d & e & f \\ g & h & i \end{pmatrix} $

% normalbasis R2 und R3 
% berechnungen zur basis

\subsection{Determinante}

%dreiecks matrizen

\subsubsection{Satz von Sarrus}
$\det(A) = aei + bfg + cdh - gec -hfa -idb.$

$\det(A) = \det \begin{pmatrix} a & b \\ c & d \end{pmatrix} = ad - cb.$

\subsection{Eigenwerte}

% Eigenwerte ermitteln:

% - mit \lambda * einheitsmatrix subtrahahieren
% - determinante ausrechnen 
% - nullstellen (von lambda) in determinanten-funktion


% vektorräume
% eigenraum
% eigenvektor
% basis

% hesse matrix

% stetigkeit
% konvergenz
% häufungspunkte
% L'Hospital

% Wahrscheinlichkeitsrechnung
% Statistik

% folgen/reihen
% Relationen

\section{Numerische Methoden}

\todo{...}

\subsection{Numerische Integration}
\todo{...}

\section{Zusätzliche Beispiele}
\todo{Beispiele}

\section{Herleitungen}
\todo{Herleitungen von Formeln}

\section{Beweise}
\todo{Beweise erklären}


\end{document}