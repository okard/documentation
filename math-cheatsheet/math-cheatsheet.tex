\documentclass[10pt,a4paper]{article}

\usepackage{ifluatex}
\ifluatex
	\usepackage{fontspec}
	\setmainfont{Linux Libertine O}
\else
	\usepackage[utf8]{inputenc}
	\usepackage{libertine}
	\usepackage[T1]{fontenc}
\fi

\usepackage[ngerman,english]{babel}
\usepackage{amsmath}
\usepackage{amsfonts}
\usepackage{amssymb}

\usepackage{color}

\newcommand{\todo}[1]{\begin{flushleft} \colorbox{yellow}{\textcolor{red}{\textbf{TODO:} {#1}}}\end{flushleft} }

\title{Mathe-Formelsammlung / Cheatsheet}

\begin{document}

\maketitle
\tableofcontents
\pagebreak


%mathe schreibweisen? dx/x


\todo{Englische Begriffe als Notiz?}
\todo{Teilgebiete der Mathematik erklären, Analysis, Algebra, ...}
\begin{flushleft}
Lizenziert unter  der Creative Commons BY-NC-SA Lizenz \\
http://creativecommons.org/licenses/by-nc-sa/4.0/
\end{flushleft}


\section{Beweisführung}
\todo{...}
\todo{Axiome}
\todo{(Bool und Logik)}

% Impliziert => <=>

% Gegenbeweise
% Induktion


\section{Zahlen}

\begin{tabular}{ll}
$\mathbb{N}$ & Ganze Zahlen 0 bis $\infty$ \\
$\mathbb{N}^*$ & Ganze Zahlen 1 bis $\infty$ \\
$\mathbb{Z}$ & Ganze Zahlen $[-\infty, \infty]$ \\
$\mathbb{Q}$ & Rationale Zahlen $\frac{m}{n}$\\
$\mathbb{R}$ & Reelle Zahlen ($\mathbb{Q}$ + Irrationale Zahlen)\\
$\mathbb{C}$ & Komplexe Zahlen\\
\end{tabular} 

\todo{Bezug auf Zahlensystem, Dezimalsystem, Zahlsystem mit Basis}

Irrationale Zahlen:
Kein Verhältnis zweier Ganzzahlen ($\frac{m}{n}$)
Beispiel: eulersche Zahl $e$, 
die Kreiszahl $\pi$, 
die Wurzel aus Zwei $\scriptstyle\sqrt{2}$ 

% unterschied(->verhältnis)  \pi mit unendlich stellen vs 0,periode 3 (1/3)

\todo{Weitere Erklärung zu Zahlengruppen/Zahlensysteme -> Referenz auf Gruppen/Ringe/etc}



\section{Mengen \& Algebraische Strukturen}

\subsection{Kommutativ, Assoziativ \& Distributiv}

${\circ}\colon A \times A\to A $\\
${\diamond}\colon A \times A\to A$ \\
$a,b,c\in A $ \\
\\
Die Verknüpfung $\circ$ ist\\
\\
\begin{tabular}{ll}
kommutativ & $a \circ b = b \circ a$ \\ 
assoziativ & $ a \circ \left( b \circ c \right) = \left( a \circ b \right) \circ c $ \\  
linksdistributiv über $\diamond$ & $a \circ (b \diamond c) = (a \circ b) \diamond (a \circ c)$ \\ 
rechtsdistributiv über $\diamond$ & $(a \diamond b) \circ c = (a \circ c) \diamond (b \circ c)$ \\ 
distributiv über $\diamond$ & wenn sie links- und rechtsdistributiv über $\diamond$ ist.
\end{tabular} 




\subsection{Mengen}
\todo{...}

\subsection{Gruppen}
Ist eine Menge
$(G,*)$\\
$*(a \in G, a \in G)$\\
$*\colon G\times G\to G,(a,b)\mapsto a*b $\\

Assoziativität: Für alle Gruppenelemente a, b und c gilt: $(a*b)*c=a*(b*c).$ \\

Es gibt ein neutrales Element $e\in G$, mit dem für alle Gruppenelemente $a\in G$ gilt: $a*e=e*a=a$.\\

Zu jedem Gruppenelement $a\in G$ existiert ein inverses Element $a^{{-1}}\in G$ mit $ a*a^{{-1}}=a^{{-1}}*a=e$.

\todo{...}

\todo{symbole: teilmenge, etc}

\subsection{Ringe}
Ist eine Menge
\todo{...}
\subsection{Körper}
Ist Menge, Gruppe
\todo{...}


\subsection{Zahlensysteme}
	Sind körper?

\subsection{Raum}

\section{Funktionen/Abbildungen/Folgen/Reihen}

\subsection{Relation}
\todo{...}

\subsection{Funktion / Abbildung}
In der Mathematik ist eine Funktion oder Abbildung eine Beziehung (Relation) zwischen zwei Mengen, 

\todo{Konvergenz/Stetigkeit/etc}


\todo{spezielle abbildungen: normen }


\subsubsection{Stetigkeit}
Nur für relle Funktionen (relevant)?  (Abbildung nach R) 

Epsilon-Delta-Kriterium

Lipschitz-Stetigkeit


\todo{Grenzwert}
	% vs folge?


\subsection{Folgen}

	Formal definiert ist eine unendliche Folge als eine Abbildung

	$a_n = x$\\
	$a_n = n$
	
	\subsubsection{Grenzwerte}
	\todo{Grenzwert}
		
		% Rechenregeln
		% Beispiele
		
	
		
	\todo{konvergent/divergent}
	\todo{monoton}
	\todo{beschränkt}
	Eine Folge heißt nach oben beschränkt, wenn sie eine obere
	 Schranke S besitzt, so dass für alle i aus  gilt: a i S. Die
	  kleinste obere Schranke einer Folge heißt auch ihr Supremum.
	   Die Begriffe nach unten beschränkt, größte untere Schranke 
	   und Infimum sind analog definiert. Eine Folge die nach oben
	    und unten beschränkt ist, heißt beschränkte Folge.
		
	Nachweis der Beschränktheit/Bestimmung einer Schrank
	Es muss also angenommen werden, dass es eine Schranke gibt
	
	
	\todo{Häufungspunkt}
	
	\todo{Konvergenzkriterium von Cauchy}
	\todo{Satz von Bolzano Weierstraß}

\todo{Relationen}

\subsection{Reihen}
	Eine Reihe ist eine Folge


\section{Komplexe Zahlen}
Spezieller Körper (zu zahlensysteme bei mengen usw?)

\renewcommand{\arraystretch}{1.5}
\begin{tabular}{ll}
Basis Definition & $ z = a + b \cdot i $ \\ 
				 & $ i^{2} = -1 $ \\ %wurzel -1 = i?
				 & $\sqrt{-1} = \sqrt{i^2} = i $ \\
				 & $Re(a + bi) = a$ \\
				 & $Im(a + bi) = b$ \\
Addition 		 & $ (a + bi) + (c + di) = (a+c)+(b+d)i $\\
Subtraktion 	 & $ (a + bi) - (c + di) = (a-c)+(b-d)i $\\
Multiplikation   &  $(a+b\,i)\cdot(c+d\,i)=(ac-bd) + (ad+bc)\cdot i. $\\
Division         & 
    $\frac{a+b\,\mathrm i}{c+d\,\mathrm i} = \frac{(a+b\,\mathrm i)(c-d\,\mathrm i)}{(c+d\,\mathrm i)(c-d\,\mathrm i)} = \frac{ac+bd}{c^2+d^2}+\frac{bc-ad}{c^2+d^2}\cdot\mathrm i. $\\
Betrag 			 & $ |z| = \sqrt{a^{2} + b^{2}} $ \\ 
\end{tabular} 

%mengen?
%logik? <-> -> etc





\section{Grund-Rechenregeln}

Dezimalsystem und $\mathbb{R}/\mathbb{C}$?
% Addition/Subtraktion/Multiplikation/Division?



\subsection{Gleichungen}
\todo{.. }

\subsection{Ungleichungen}
\todo{.. }
% tips, annahmen, etc

\subsection{Potenzen}

$ a^{b} * a^{c} = a^{b+c} $\\
$ \dfrac{1}{x^2} = x^{-2}$\\
$a^{r-s}=\frac{a^r}{a^s}$\\
$ a^{-r} = \frac{1}{a^r}$\\

\subsection{Wurzeln}

$\sqrt[n]{a} \cdot \sqrt[n]{b} = \sqrt[n]{a \cdot b}  $ \\
$\sqrt[n]{x^{m}} = x^{\frac{m}{n}}  $ \\
$\sqrt[n]{x} = x^{\frac{1}{n}} = exp(\dfrac{ln(x)}{n})$

\subsection{Brüche}

\todo{Addition}
\todo{Subtraction}
\todo{Multiplikation}
\todo{Division}

$ \dfrac{a^{n}}{b^{m}} = a^{n} \cdot b^{-m}$

\subsection{Summen}

\todo{Generell}

$ \sum_{k=m}^{n}a_k = \sum_{m \leq k \leq n}a_k = a_m + a_{m+1} + \dots + a_n $

\todo{Umformungen}

% elemente rausziehen
% aufteilen

\subsection{Fakultät}

$ n! = 1\cdot 2\cdot 3\cdot \dots\cdot n =  \prod \limits_{k=1}^{n}k $\\
$ n! = n * (n-1)!$\\
$ 0! = 1 $\\
$ 1! = 1 $\\

\todo{Umformungen}

Einzelne Faktoren die nicht vom Index abhängen lassen sich vorziehen:
$\prod \limits_{k=1}^{n} a* k  = a^n * \prod \limits_{k=1}^{n} k$


% n! =n *(n -1)! 
% \(\prod \limits_{i=1}^{n+1}i = 1\cdot 2\cdot\dots\cdot n\cdot (n+1) \)
%binomialkoeffizient


\subsection{e-Funktion/Logarithmus}

$ e = \lim_{n\to\infty} \left(1+\frac{1}{n}\right)^n $\\
$ e^{x} = exp(x) = \sum_{n=0}^\infty \dfrac{x^{n}}{n!} $ \\
$ e^{1} = \sum_{n=0}^\infty \dfrac{1}{n!} = 2.7182818284590...$\\
$ a^{x} = e^{x \cdot ln a}$\\

$ ln $ natürlicher logarithmus $ e^x \rightarrow x = log_e (e^x) = ln(e^x)$ \\
$ ln(e^y) = y $\\
$ ln(\frac{1}{a}) =  $\\

$ log_b a $ 

$ lg lb $\\
$ \log_b \frac 1x = -\log_b x $


\subsection{Trigonometrische Funktionen (sin/cos/tan)}


\todo{Bogenmaß}
\todo{Einheitskreis}
% bogenmaß
% umfang des einheitskreis -> 2\pi
% bogenmaß = [0, 2\pi]
% rad
% der grad ist dann 0-360 <-> 0 - 2\pi

% tabelle Grad-PI-rad sin/cos siehe script
\todo{Tabelle Grad/PI/RAD -> cos/sin/tan}

%Winkel \alpha (Grad) 	0^\circ 	30^\circ 	45^\circ 	60^\circ 	90^\circ 	180^\circ 	270^\circ 	360^\circ
%Bogenmaß 	0 	\frac{\pi}{6} 	\frac{\pi}{4} 	\frac{\pi}{3} 	\frac{\pi}{2} 	\pi 	\frac{3\pi}{2} 	2\pi
%Sinus 	\frac12\sqrt0 = 0 	\frac12\sqrt1 = \frac12 	\frac12\sqrt2=\frac{1}{\sqrt2} 	\frac12\sqrt3 	\frac12\sqrt4 = 1 	0 	-1 	0
%Kosinus 	\frac12\sqrt4 = 1 	\frac12\sqrt3 	\frac12\sqrt2=\frac{1}{\sqrt2} 	\frac12\sqrt1 = \frac12 	\frac12\sqrt0 = 0 	-1 	0 	1
% tangens


%umformungen sinus auflösen etc

\todo{Umformungen}

\subsection{Binominialkoeffizient}

$ \binom nk $
$ \binom nk = \prod_{j=1}^k \frac{n + 1 - j}j$

%rechenregeln




\section{Hilfsmittel}

% für umformungen?

\begin{itemize}
\item Binomische Formeln
\item Quadratische Ergänzung
\item Vollständige Induktion
\end{itemize}

% quadratische Ergänzung
% binomische formeln

%% einordnen von gleichungen?


\section{Vektoren}

\subsection{Notation}
$\mathbb{R}^{n}$
$\vec{v}$

\subsection{Rechenregeln}
%rechnen add/sub/mul/div
% 

\section{Matrizen}

\subsection{Notation}

Beispiel für Matrix-Raum $\mathbb{R}^{m\times n}$ das Bedeutet jede Matrix aus diesem Raum hat $m \cdot n $ Elemente das sind $m$ Zeilen und $n$ Spalten.
Ein Element kann referenziert werden mit $a_{ij}$ mit $i \in [1, m], j \in [1, n] $
\\

$A \in \mathbb{R}^{3\times 3} = \begin{pmatrix} a & b & c \\ d & e & f \\ g & h & i \end{pmatrix} 
   = \begin{pmatrix} a_{11} & a_{12} & a_{13} \\ a_{21} & a_{22} & a_{23} \\ a_{31} & a_{32} & a_{mn} \end{pmatrix} 
 $



\subsubsection{Identitäts-Matrix}

% normalbasis R2 und R3 
% berechnungen zur basis


\subsection{Matrix Operationen}

\subsubsection*{Addition / Subtraktion}

Beide Matrizen müssen vom selben Typ sein also z.B. $A,B \in \mathbb{C}^{m\times n}$

$ A+B := (a_{ij}+b_{ij})_{i=1 , \ldots , m; \ j=1 , \ldots , n} $


$\begin{pmatrix} 1 & -3  \\ 1 & 2  \end{pmatrix} 
+ \begin{pmatrix} 0 & 3  \\ 2 & 1 \end{pmatrix} 
= \begin{pmatrix} 1+0 & -3+3  \\ 1+2 & 2+1  \end{pmatrix} 
= \begin{pmatrix} 1 & 0  \\ 3 & 3  \end{pmatrix} 
$

Subtraktion äquialent dazu

\subsubsection*{Skalar-Multiplikation}


$\lambda\cdot A := (\lambda\cdot a_{ij})_{i=1, \ldots , m; \ j=1, \ldots , n}$

$\lambda\cdot \begin{pmatrix} a & b  \\ c & d  \end{pmatrix} = 
\begin{pmatrix} a \cdot\lambda & b\cdot\lambda  \\ c\cdot\lambda & d\cdot\lambda  \end{pmatrix} $

\subsubsection*{Matrix-Multiplikation}


$B \cdot A \neq A \cdot B $
aber:
$(A \cdot B) \cdot C = A \cdot (B \cdot C) $

\subsubsection*{Transponiert}

\subsubsection*{Inverse}

\subsubsection*{Multiplikation mit Vektoren}


\subsection{Eigenschaften}
% hermitesch
% diagonalisierbar

\subsection{Determinante}

%dreiecks matrizen

\subsubsection{Satz von Sarrus}
$\det(A) = aei + bfg + cdh - gec -hfa -idb.$

$\det(A) = \det \begin{pmatrix} a & b \\ c & d \end{pmatrix} = ad - cb.$

\subsection{Eigenwerte}

% Eigenwerte ermitteln:

% - mit \lambda * einheitsmatrix subtrahahieren
% - determinante ausrechnen 
% - nullstellen (von lambda) in determinanten-funktion


\todo{vektorräume}
\todo{eigenraum}
\todo{eigenvektor}
\todo{ basis }

\todo{hesse matrix}

\todo{häufungspunkte}
\todo{L'Hospital}



\section{Differentialrechnung}

% ACHTUNG vs Differentialgleichung DGL
Bei Differentialrechnung Berechnung lokaler Veränderungen von Funktionen. Steigung an bestimmten Punkten.

\subsection{Notation}
\begin{flushleft}

Eine Funktion $f \colon U \to \mathbb{R}$, die ein offenes Intervall $U$ in die reellen Zahlen abbildet, heißt differenzierbar an der Stelle $x_0 \in U$, falls der Grenzwert existiert.

\[\lim_{x\to x_0} \frac{f(x) - f(x_0)}{x - x_0} = \lim_{h\to 0} \frac{f(x_0 +h) - f(x_0)}{h}   (h = x - x_0)\]

%steigungsdreieck?

Dieser Grenzwert heißt Differentialquotient oder Ableitung von $f$ nach $x$ an der Stelle $x_0$ und wird als
$ f'(x_0)$   oder   $\left.\frac{\mathrm df(x)}{\mathrm dx}\right|_{x=x_0}$   oder   $\frac{\mathrm df}{\mathrm dx}(x_0)$   oder   $\frac{\mathrm d}{\mathrm dx}f(x_0)$   notiert 

Nicht überall differenzierbar:
$f(x) = |x|$

$f'(0)$ existiert nicht

\end{flushleft}
    
\subsection{Ableitungsregeln}


\begin{tabular}{ll}
Potenzregel & $\left(x^n\right)' = n x^{n-1} $ \\
Produktregel & $(g\cdot h)' = g' \cdot h + g \cdot h'$ \\
Quotientenregel & 
    $\left(\frac{g}{h}\right)' = \frac{g' \cdot h - g \cdot h'}{h^2}$ \\
Kettenregel & $(g \circ h)'(x) = (g(h(x)))' = g'(h(x))\cdot h'(x)$\\
\end{tabular}

% (\ln(f))' = \frac{f'}{f} 


\subsubsection{Nullstellen}


\todo{p/q-formel}
%nullstellen x² x³ -> p/q formel

$ x^2+px+q=0$\\
$x_{1,2} = - \frac{p}{2}\pm\sqrt{\left(\frac{p}2\right)^2 - q}$

%Kurveldiskussion
% sattelpunkt
% extrem punkt

% differenzierung ableitungsregeln

\subsection{Partielle Ableitungen}

Für Funktionen mit mehreren Argumenten:
$f(x,y) = x^2 \sin(xy)$

Für einzelne Argumente ableiten:
$\dfrac{\partial f}{\partial x_i}$

% jacobi matrix

%mittelwertsatz der differntialrechnung

% rechnen mit mehreren variablen


\section{Integralrechnung}

Bei Integralrechnung Berechnung von einer Fläche.

Umkehrschritt der Differentialrechnung
Aufleitung

Berechnet Fläche einer Funktion
\todo{[Bild]}


\subsection{Stamm-Funktionen}

% https://www.youtube.com/watch?v=8ynwPeV_VGQ

%wurzel oder e funktion -> substituieren

% substitution
\subsection{Substitution}

% https://www.youtube.com/watch?v=z_flzV7SBBw

\subsection{Partielle Integration} 

% partielle Integration, Substitution

$\int f'(x) * g(x) dx = f(x)*g(x) - \int f(x) * g'(x) dx$

%partialbruch zerlegung

%mittelwertsatz der integralrechnung

% rechnen mit mehreren variablen

\section{Differentialgleichungen}

Bei Differentialgleichungen(DGL) Funktion gesucht die von einer oder mehreren Variablen abhängt und in der auch Ableitungen dieser Funktion vorkommen.


\todo{Schreibweisen}

\subsection{Klassifikation}


\subsubsection{Ordnung}
	Die Ordnung entspricht der maximalen Ableitung

\subsubsection{Explizit/Implizit}
	Explizit: Die höchste Ableitung steht auf einer Seite alleine.
	
	
\subsubsection{Gewöhnliche/Partielle}
	Eine unabhänige variable ist gewöhnlich
	$f(x)$

	Mehrere Unabhängige Variablen?
	$f(x,y)$
	Partielle Ableitungen
	
	
\subsubsection{Linear/Nicht linear}

	$sin(x) \cdot y'' + 42 \cdot y' - x^2 \cdot y = $
	
	$ X \cot y $
	y einfach, nicht als Quadrat/im Sinus/Unter Wurzel/ etc
	
	X funktionen von unabhängigenvariable x, konstante, ...
	
	(imhomoginität darf von X abhängen)
	(-> Lineare Gleichungssysteme)
	
	Mit Konstanten-Koeffizienten
	
	
	Homogene Lineare 
	ist was anderes als generell homogen/inhomogen
	



Varianten:
	(DGL)
	einfache Differentialgleichungen
	partielle Differentialgleichungen

	homogone autonome
	
	lineare
	
Schreibweisen:
	$ f'(x) = 3f(x) + x^2 $
	$ y'(x) = 3y(x) + x^2 $
	



\section{Numerische Methoden}

\todo{...}
\todo{IT-Bezug}

\subsection{Numerische Integration}
\todo{...}


\section{Wahrscheinlichkeitsrechnung}
\todo{...}
\todo{Statistik}

\section{Statistik}


\section{Zusätzliche Beispiele}
\todo{Zahlentheorie asymmetrische Krytografie}
\todo{Beispiele}

\section{Herleitungen}
\todo{Herleitungen von Formeln}

\section{Beweise}
\todo{Beweise erklären}

\todo{Umformungsbeispiele?}


\end{document}